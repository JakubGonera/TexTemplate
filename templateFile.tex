%Wpisz się jako autor (wyświetla się na co drugiej stronie w nagłówku)
\renewcommand{\autor}{Jan Kowalski}
%Twoja praca powinna zawierać się w jednym rozdziale, dlatego nazwij go aby jak najlepiej opisywał temat jaki wybrałeś (nazwa wyświetla się w nagłówku)
\chapter{Lorem ipsum}
%Tu się podpisz np. opracowane przez XYZ
\textit{Najlepiej się tu podpisać, jak chcesz to napisz też dedykację.}
%reszta pliku to przykładowe użycie, pamiętaj że sekcje i subsekcje znajdą się w spisie treści
\section{Jakaś sekcja}
Lorem ipsum dolor sit amet, consectetur adipiscing elit. Aenean eleifend augue et sem hendrerit varius. Morbi id sodales lectus, a imperdiet odio. Cras at bibendum augue, in convallis leo. Vestibulum et sem eget tellus tincidunt varius vel ut quam. Curabitur vel velit aliquet, aliquam metus ac, fringilla eros. Nam rutrum tincidunt mauris sed congue. Integer vehicula nulla sit amet odio scelerisque elementum. Fusce ex mauris, scelerisque vitae massa eu, ullamcorper sodales enim. Vestibulum consequat sit amet neque tempor lobortis. Donec eu augue nulla.
\section{Inna sekcja}
Vivamus pretium nunc dui. Maecenas in auctor lacus. In hac habitasse platea dictumst. Nullam maximus tincidunt purus sed convallis. Fusce nec pharetra tortor, non tincidunt nunc. Mauris eu convallis tellus. Aenean ac mauris vel odio iaculis dictum. Integer et consequat eros, sed lacinia diam. Sed auctor posuere ligula, ut fringilla urna scelerisque eu.
\subsection{Przykładowa podsekcja}
Morbi dictum nulla metus, ut malesuada velit tempus ac. Mauris nulla eros, euismod et neque ultricies, dignissim rhoncus turpis. Suspendisse tempus sit amet libero id viverra. Vivamus congue dictum erat et vestibulum. Praesent bibendum suscipit leo eu scelerisque. Nunc pellentesque, dui eu tristique sagittis, risus tellus aliquam dui, nec vulputate ipsum nibh id libero. Curabitur blandit massa sit amet magna porttitor pretium. Etiam feugiat pretium lacus vel vehicula. Mauris interdum lacus at ligula convallis scelerisque. Nullam quis odio felis. Nullam posuere, eros viverra sollicitudin molestie, lorem massa accumsan odio, ac vehicula elit eros vel dolor. Vestibulum aliquet condimentum nisl ac suscipit.
\section{Vivamus congue dictum erat et vestibulum}
Maecenas sem mauris, placerat in scelerisque non, efficitur at mi. Ut eget dui enim. Pellentesque et risus vel diam gravida congue lobortis nec neque. Pellentesque habitant morbi tristique senectus et netus et malesuada fames ac turpis egestas. Pellentesque faucibus risus libero, non sollicitudin est fringilla sed. Donec velit augue, aliquam vel tincidunt ut, maximus ac est. Sed luctus interdum nulla id vehicula. Vivamus at pretium erat. Morbi dapibus turpis tortor, id laoreet elit maximus id. Praesent a vulputate dui. Nunc eu tortor et mauris vestibulum pharetra et vitae augue. Ut elementum tellus sit amet nisi congue, sit amet consequat arcu tincidunt. In blandit at ligula sit amet lobortis. In sed nibh quis diam accumsan condimentum. Suspendisse eget nisi magna. Donec pretium libero nec scelerisque pretium.
\begin{theorem}
    Każda liczba naturalna posiada dzielnik pierwszy.
\end{theorem}
\begin{proof}
    Vivamus at pretium erat. Morbi dapibus turpis tortor, id laoreet elit maximus id. Praesent a vulputate dui. Nunc eu tortor et mauris vestibulum pharetra et vitae augue. Ut elementum tellus sit amet nisi congue, sit amet consequat arcu tincidunt.
\end{proof}
\newpage
Vivamus pretium nunc dui. Maecenas in auctor lacus. In hac habitasse platea dictumst. Nullam maximus tincidunt purus sed convallis. Fusce nec pharetra tortor, non tincidunt nunc. Mauris eu convallis tellus. Aenean ac mauris vel odio iaculis dictum. Integer et consequat eros, sed lacinia diam. Sed auctor posuere ligula, ut fringilla urna scelerisque eu.
\begin{definition}
    Lorem ipsum dolor sit amet, consectetur adipiscing elit.
\end{definition}
\begin{remark}
    Lorem ipsum dolor sit amet, consectetur adipiscing elit.
\end{remark}

\section{Zadania}
\begin{problem}
    \label{zadanie1}
    Lorem ipsum dolor sit amet, consectetur adipiscing elit. Aenean eleifend augue et sem hendrerit varius.
    \begin{equation*}
        \sum^{\infty}_{i} = 2^i
    \end{equation*}
    Morbi id sodales lectus, a imperdiet odio. Cras at bibendum augue, in convallis leo.
\end{problem}
\begin{problem}
    \label{zadanie2}
    Lorem ipsum dolor sit amet, consectetur adipiscing elit. Aenean eleifend augue et sem hendrerit varius.
    \begin{equation*}
        \sum^{\infty}_{i} = 2^i
    \end{equation*}
    Morbi id sodales lectus, a imperdiet odio. Cras at bibendum augue, in convallis leo.
\end{problem}

\section{Wskazówki i rozwiązania}
\textbf{Zadanie \ref{zadanie1}}
Lorem ipsum dolor sit amet, consectetur adipiscing elit. Aenean eleifend augue et sem hendrerit varius.
\newline
\textbf{Zadanie \ref{zadanie2}}
Lorem ipsum dolor sit amet, consectetur adipiscing elit. Aenean eleifend augue et sem hendrerit varius.